\documentclass{article}
\usepackage{amsmath}
\usepackage{amssymb}
\usepackage{bm}
\usepackage{graphicx}
\usepackage{epstopdf}
\DeclareGraphicsRule{.tif}{png}{.png}{`convert #1 `basename #1 .tif`.png}
\usepackage{color}
\pagestyle{plain}
%\pagestyle{empty}
\textheight 9 true in
\textwidth 6.5 true in
\hoffset -.75 true in
\voffset -.75 true in
 
\mathsurround=2pt  \parskip=2pt
\def\crv{\cr\noalign{\vskip7pt}} 
\def\a{\alpha } \def\b{\beta } \def\d{\delta } \def\D{\Delta } \def\e{\epsilon }
\def\g{\gamma } \def\G{\Gamma} \def\k{\kappa} \def\l{\lambda } \def\L{\Lambda }
\def\th{\theta } \def\Th{\Theta} \def\r{\rho} \def\o{\omega} \def\O{\Omega}
\def\ve{\varepsilon} 

\def\sA{{\cal A}} \def\sB{{\cal B}} \def\sC{{\cal C}} \def\sI{{\cal I}}
\def\sR{{\cal R}} \def\sF{{\cal F}} \def\sG{{\cal G}} \def\sM{{\cal M}}
\def\sT{{\cal T}} \def\sH{{\cal H}} \def\sD{{\cal D}} \def\sW{{\cal W}}
\def\sL{{\cal L}} \def\sP{{\cal P}} \def\s{\sigma } \def\S{\Sigma}
\def\sU{{\cal U}} \def\sV{{\cal V}} \def\sY{{\cal Y}}

\def\gm{\gamma -1}
\def\summ{\sum_{j=1}^4}

\def\bb{{\bm b}} \def\yb{{\bm y}}
\def\ub{{\bm u}}  \def\xb{{\bm x}} \def\vb{{\bm v}} \def\wb{{\bm w}}
\def\omegab{{\bm \omega}} \def\rb{{\bm r}} \def\ib{{\bm i}} \def\jb{{\bm j}}
\def\lb{{\bm l}} \def\kb{{\bm k}} \def\Ab{{\bm A}} \def\fb{{\bm f}} \def\Ub{{\bm U}}
\def\Fb{{\bm F}} \def\nb{{\bm n}} \def\Db{{\bm D}} \def\eb{{\bm e}}
\def\gb{{\bm g}}  \def\Gb{{\bm G}} \def\hb{{\bm h}} \def\Yb{{\bm Y}} \def\Rb{{\bm R}} 
\def\Tb{{\bm T}}

\def\As1{{\bf {\cal A}}_1}\def\DO{{\cal D}_0} \def\UO{{\cal U}_0}
\def\ie{{\it{i.e.}}}

\def\ubbar{{\bf {\bar{u}}}} \def\sbar{{\bar{\sigma }}} \def\ubar{{\bar{u}}}  
\def\abar{{\bar{a}}} \def\vbar{{\bar{v}}}  \def\rbar{{\bar{\rho}}}
\def\pbar{{\bar{p}}} \def\ebar{{\bar{e}}} \def\Tbar{{\bar{T}}}
\def\bbar{{\bar{\beta}}} \def\Mbar{{\bar{M}}}  \def \sMbar{{\bar{\cal M}}}
\def\Ebar{{\bar{E}}} \def\sMbar{{\bar{\cal M}}}
\def\sPbar{{\bar{\cal P}}} \def\xbar{{\bar{x}}}

\newcommand{\pdv}[2]{\frac{\partial#1}{\partial#2}}
\newcommand{\dv}[2]{\frac{d#1}{d#2}}
\newcommand{\ord}[2]{#1^{(#2)}}
\newcommand{\vct}[1]{\vec{#1}}

 \newcommand{\bc}{\begin{center}}
 \newcommand{\ec}{\end{center}}
 
 \newcommand{\bq}{\begin{equation}}
 \newcommand{\eq}{\end{equation}}
 
 \newcommand{\beqs}{\begin{eqnarray}}
 \newcommand{\eeqs}{\end{eqnarray}}
 
 \newcommand{\beqa}{\begin{eqnarray*}}
 \newcommand{\eeqa}{\end{eqnarray*}}
 
 \newcommand{\ol}{\overline}
 \newcommand{\ul}{\underline}
 
 \newcommand{\dint}{{\int \!\! \int \!\!}}
 \newcommand{\tint}{{\int \!\! \int \!\! \int \!\!}}
 
 \newcommand{\bfig}{\begin{figure}}
 \newcommand{\efig}{\end{figure}}
 
 \newcommand{\cen}{\centering}
 \newcommand{\n}{\noindent}
 
 \newcommand{\btab}{\begin{table}}
 \newcommand{\etab}{\end{table}}
 
 \newcommand{\btbl}{\begin{tabular}}
 \newcommand{\etbl}{\end{tabular}}
 
 \newcommand{\bdes}{\begin{description}}
 \newcommand{\edes}{\end{description}}
 
 \newcommand{\benum}{\begin{enumerate}}
 \newcommand{\eenum}{\end{enumerate}}
 
 \newcommand{\bite}{\begin{itemize}}
 \newcommand{\eite}{\end{itemize}}
 
 \newcommand{\cle}{\clearpage}
 \newcommand{\npg}{\newpage}
 
 \newcommand{\bss}{\begin{singlespace}}
 \newcommand{\ess}{\end{singlespace}}
 
 \newcommand{\bhalf}{\begin{onehalfspace}}
 \newcommand{\ehalf}{\end{onehalfspace}}
 
 \newcommand{\bds}{\begin{doublespace}}
 \newcommand{\eds}{\end{doublespace}}
 
 \newcommand{\eps}{\mbox{$\epsilon$}} 
 \newcommand{\stilde}{\mbox{$\tilde s$}} 
 \newcommand{\shat}{\mbox{$\hat s$}} 

 \newcommand{\blue}{\color{blue}}
 \newcommand{\red}{\color{red}}
 \newcommand{\magenta}{\color{magenta}}
 \newcommand{\green}{\color{green}}
 \newcommand{\nc}{\normalcolor}




\pagestyle{empty}
\begin{document}

\begin{center}
\large{ MATH-6620 \hspace{1in}  PERTURBATION METHODS \hspace{1in}SPRING 2016\\ Homework-4 \\ Assigned Wednesday February 24, 2016 \\ Due Wednesday March 30, 2016}\end{center}

\vspace{6 ex}

{\bf Name: Michael Hennessey} \hfill

\vspace{6 ex}

\bc {\bf PROBLEMS} \ec

\benum

% problem 1
\item Consider the signaling problem
\begin{equation*}
\e(u_{xx} - u_{tt}) = u_t + 2 u_x, \quad 0<x<\pi, \; t>0,
\end{equation*}
with auxiliary conditions
\begin{equation*}
u(x,0) = u_t(x,0) = 0, \;\; u(0,t) = -\sin t, \; u(\pi,t) = 0.
\end{equation*}
Construct a leading-order solution for $0 < \e \ll 1,$ paying due attention to the location of any layers.\\

Solution:\\

We begin by finding the outer solution of the PDE by letting $u(x,t;\e)\sim u_0(x,t)$ and collecting only the $O(1)$ terms. This gives the linear transport equation
$$u_{0t}+2u_{0x}=0.$$
This equation has characteristics
$$\gamma=x-2t$$
which implies the solution to the PDE is then
$$u_0=A(\gamma)=A(x-2t).$$
We now apply the auxiliary conditions to find a reasonable outer solution. Clearly, if we apply the boundary conditions at $t=0$, we get the zero solution. Similarly, we get the zero solution if we apply the condition at $x=\pi$. However, if we apply the boundary condition at $x=0$ we get an interesting solution.
$$u_0(0,t)=A(-2t)=-\sin t\implies u_0(x,t)=\sin(\frac{x-2t}{2}).$$
Thus if we let $u_0$ be a piecewise continuous function defined
$$u_0(x,t)=\left\{\begin{array}{cc}0,&x\geq 2t\\ \sin(\frac{x-2t}{2}),&x<2t\end{array}\right.$$
we satisfy every boundary condition but the one at $x=\pi$ and remove any possibility of a leading-order layer along $x=2t$. However, ince the line $x=\pi$ is transverse to the subcharacteristics of the equation,  we do find that there is an $\e$-thick layer at $x=\pi, t>\pi/2.$ Note we are only concerned with a layer at $t>\pi/2$ for the boundary condition at $x=\pi$ is only dissatisfied on the right side of the line $x=2t.$ Thus we let
$$x=\pi+\e\xi,\;\;t=T+\frac{\pi}{2},\;T>0,\;\;u(x,t)=U(\xi,T).$$
Then the original signalling problem becomes
$$U_{0\xi\xi}=2U_{0\xi},\;\;U(0,T)=0$$
in the layer. The inner solution is then
$$U_0(\xi,T)=\frac{1}{2}B_0(T)(e^{2\xi}-1).$$
We then match to the outer solution in this region. We let $\xi\to\infty$ in $U_0$ to find
$$U_0(\xi,T)\to-\frac{1}{2}B_0(T),$$
and let $x\to \pi$ in $u_0$ to find
$$u_0(x,t)\to \sin(\frac{\pi-2t}{2}).$$
Thus we express the outer solution in the inner variable and we find that
$$u_0\to -\sin(T)=-\frac{1}{2}B_0(T)\implies B_0(T)=2\sin(T).$$
Then the inner solution is
$$U_0(x,t)=\cos(t)\left(1-e^{2(x-\pi)/\e}\right).$$
To then write the composite solution, we first note that the common part found above
$$-\sin(T)=\cos(t).$$
Now we simply add the inner and outer solutions together and subtract off a $\cos(t)$ to find
$$u_C(c,t)=\left\{\begin{array}{cc}0,&x\geq 2t\\ \sin\left(\frac{x-2t}{2}\right)-\cos(t)e^{2(x-\pi)/\e},&x<2t\end{array}\right..$$


% Problem 2
\item Consider the elliptic problem
\begin{equation*}
\e(u_{xx} + u_{yy}) + u_x + u_y + u = 0, \quad 0<x<1, \; 0<y<1,
\end{equation*}
with boundary conditions
\begin{equation*}
u(x,0) = 0, \; u(x,1) = 1-x, \;\; u(0,y) = e^{-y}, \; u(1,y) = 1-y.
\end{equation*}
\benum
\item Construct a leading-order solution for $0 < \e \ll 1,$ paying due attention to the location of the layers.

\item Repeat the problem if the second boundary condition above is changed to $u(x,1) = 1.$

\eenum
Solution:\\

\benum
\item We first determine the subcharacteristic of the PDE by looking at the outer $O(1)$ equation
$$u_x+u_y+u=0\implies y=x+\gamma\implies u=A(\gamma)e^{-x}=A(y-x)e^{-x}.$$
We then satisfy the boundary conditions on the right and upper boundaries to get the outer solution
$$u_0(x,t)=\left\{\begin{array}{cc}(y-x)e^{1-y},&x<y\\(x-y)e^{1-x},&y<x\end{array}\right.$$
We choose to satisfy these boundary conditions (and therefore get backward flowing characteristics) because this outer solution is continuous along $x=y$ suggesting that there is no leading-order layer at $x=y.$ This outer solution does lead us to beleive that there are $O(\e)$ thick layers at $x=0$ and $y=0.$ Thus we derive and solve the inner PDEs:
\benum
\item We let $x=\e\xi,$ and $u(x,y;\e)=U(\xi,y;\e).$ Collecting the $O(1)$ terms and letting $U(\xi,y;\e)\sim U_0(\xi,y)$ gives the PDE
$$U_{0\xi\xi}=-U_{0\xi},$$
with solution
$$U_0(\xi,y)=-A_0(y)e^{-\xi}+B_0(y).$$
Since we must satisfy $u(0,y)=e^{-y}$ we have
$$U_0(0,y)=-A_0(y)+B_0(y)=e^{-y}\implies B_0=A_0(y)+e^{-y}$$
giving us the inner solution
$$U_0(\xi,y)=A_0(y)(1-e^{-\xi})+e^{-y}.$$
To determine $A_0(y)$ we must match the inner and outer solutions in the layer. We let $\xi\to\infty$ and $x\to 0$ in the inner and outer solutions respectively to find
$$U_0(\xi,y)\to A_0(y)+e^{-y}\;\;u_0(x,y)\to ye^{1-y}.$$
Hence we determine that
$$A_0(y)=ye^{1-y}-e^{-y}\;\implies\;U_0(x,y)=ye^{1-y}-ye^{1-y-x/\e}+e^{-y-x/\e}.$$
We now form the left composite solution:
$$u_{CL}=(y-x)e^{1-y}-ye^{1-y-x/\e}+e^{-y-x/\e}.$$

\item Now we determine the equation in the layer at $y=0.$ We let $y=\e\eta$, and $u(x,y;\e)=v(x,\eta;\e)$ and find that the leading order equation in the layer is
    $$v_{0\eta\eta}=-v_{0\eta},\;\;v_0(x,0)=0.$$
    This equation has solution
    $$v_0(x,\eta)=C_0(x)(1-e^{-\eta}).$$
    We then match to determine $C_0(x).$ We let $\eta\to\infty$ in the inner solution and $y\to 0$ in the outer solution to find
    $$v_0\to C_0(x),\;\;u_0\to xe^{1-x}.$$
    Thus we have
    $$C_0(x)=xe^{1-x}.$$
    The inner solution can then be written
    $$v_0(x,y)=xe^{1-x}(1-e^{-y/\e}).$$
    We then form the composite solution on the right of the center characteristic line
    $$u_{CR}=(x-y)e^{1-x}-xe^{1-x-y/\e}.$$
    Note that we do see a transcendentally small error at $u(1,y)$ with this composite.
    The full composite can be written
    $$u_c=\left\{\begin{array}{cc}(y-x)e^{1-y}-ye^{1-y-x/\e}+e^{-y-x/\e},&x<y\\(x-y)e^{1-x}-xe^{1-x-y/\e},&y<x\end{array}\right..$$
    \eenum

\item The problem changes considerably when we replace the boundary condition as stated above. We no longer have a piecewise continuous outer solution, which therefore implies we have a boundary layer on the characteristic line $x=y.$ In determining the outer solution, we satisfy the same boundary conditions as before, except the new boundary condition results in the solution
    $$u(x,y)=\left\{\begin{array}{cc}e^{1-y},&x<y\\(x-y)e^{1-},&x>y\end{array}\right..$$
    Thus we only satisfy the boundary conditions along the top of the domain and the right of the domain. We can see along the line $x=y$ this outer solution is not continuous. Thus we now have a third layer. We will begin by finding the inner solutions along the left and bottom sides of the domain.
    \benum
    \item First we let $x=\e \xi$ and $u(x,y;\e)=U(\xi,y;\e)$ then collect the resulting $O(1)$ terms in the PDE given by the transformed variables to get
        $$U_{0\xi\xi}=-U_{0\xi},\;\;U_0(0,y)=e^{-y}.$$
        This equation has the solution
        $$U_0(\xi,y)=A_0(y)(1-e^{-\xi})+e^{-y}.$$
        We then match the inner and outer solutions as $\e\to 0$ to find that
        $$e^{1-y}=A_0(y)+e^{-y}\implies A_0(y)=e^{1-y}-e^{-y}.$$
        Thus we know the inner solution is
        $$U_0(x,y)=e^{1-y}-e^{1-y-x/\e}+e^{-y-x/\e}.$$
        The inner solution is also the composite solution on the left:
        $$u_{cl}=e^{1-y}+e^{1-y}-e^{1-y-x/\e}+e^{-y-x/\e}-e^{1-y}=e^{1-y}-e^{1-y-x/\e}+e^{-y-x/\e}.$$
        The solution near the layer on the bottom of the domain is the same as before
        $$u_{cr}=(x-y)e^{1-x}-xe^{1-x-y/\e}.$$
    \item Now we inspect the layer along $x=y$. As the layer is parallel to the characteristics, we know it has thickness $O(\sqrt{\e}).$ Thus we let $x=y+\sqrt{\e}\eta$, and $u(x,y;\e)=v(\eta,y;\e)$ and we get the PDE
        $$2v_{\eta\eta}-2\sqrt{\e}v_{\eta y}+\e v_{yy}+v_y+v=0.$$
        Then if we let $v(\eta,y;\e)\sim v_0(\eta,y)$ we find the $O(1)$ PDE in the layer behaves like a reverse diffusion with source:
        $$2v_{0\eta\eta}+v_{0y}+v_0=0.$$
        We can use the tranformation $v_0(\eta,y)=e^{1-y}w_0(\eta,y)$ to get a simpler equation
        $$w_{0\eta\eta}=\frac{-1}{2}w_{0y}.$$
        We solve this equation by letting
        $$w=f(\gamma)=f\left(\frac{-i\eta}{\sqrt{y}}\right).$$
        This results in the differential equation for $\gamma$
        $$f''(\gamma)=-\frac{\gamma}{4}f'(\gamma).$$
        This equation has solution
        $$f\left(\frac{-i\eta}{\sqrt{y}}\right)=\sqrt{2\pi}c_1 erf\left(\frac{-i\eta}{2\sqrt{2 y}}\right)+c_2=w_0(\eta,y).$$
        $$\implies v_0(\eta,y)=e^{1-y}\sqrt{2\pi}c_1 erf\left(\frac{-i\eta}{2\sqrt{2 y}}\right)+c_2e^{1-y}.$$
        We then write the inner solution in terms of the outer variable to find
        $$v_0(x,y)=e^{1-y}\sqrt{2\pi}c_1 erf\left(\frac{i(y-x)}{2\sqrt{2y \e}}\right)+c_2 e^{1-y}.$$
        Now if we let $\e\to 0$ we get
        $$v_0\to \left\{\begin{array}{cc}c_1 \infty +c_2 e^{1-y},&x<y\\ -c_1 \infty +c_2 e^{1-y}, &\end{array}\right.$$
        Now we rewrite the leading order outer solution in terms of the inner variable:
        $$u_0(\eta,y)=\left\{\begin{array}{cc}e^{1-y},&x<y\\ \sqrt{\e}\eta e^{1-y-\sqrt{\e}\eta},&x>y\end{array}\right.$$
        then we take the limit as $\e\to 0$ to find
        $$u_0\to\left\{\begin{array}{cc}e^{1-y},&x<y\\0&x>y\end{array}\right..$$
        Then clearly matching fails in this case, unless we take the trivial inner solution
        $$v_0=e^{1-y}.$$
        Instead, we must find some other way to match. The only method I could find that works is to take the modulus of the argument of the error function located in the inner solution. Perhaps there was a mistake in our derivation of the PDEs in the inner layer or their solutions. The other possibility is that the backwards diffusion equation's initial conditions require the solution be a real error function instead of an imaginary error function. Either way, we proceed by adjusting our inner solution to
        $$v_0=e^{1-y}\sqrt{2\pi}c_1 erf\left(\frac{(y-x)}{2\sqrt{2y \e}}\right)+c_2 e^{1-y}.$$
        Then when we limit $\e\to 0$ we get
        $$v_0\sim \left\{\begin{array}{cc}e^{1-y}(\sqrt{2\pi}c_1+c_2),&x<y\\ e^{1-y}(c_2-\sqrt{2\pi c_1}), &x>y\end{array}\right..$$
        Then matching gives us $c_1=1/(2\sqrt{2\pi})$ and $c_2=1/2$. Thus we have the matched inner solution
        $$v_0(x,y)\frac{1}{2}e^{1-y}\left(erf\left(\frac{y-x}{2\sqrt{2\e y}}\right)+1\right).$$
        Now we form the composite solution:
        $$u_c=\left\{\begin{array}{cc}-e^{1-y-x/\e}+e^{-y-x/\e}+\frac{1}{2}\left(erf\left(\frac{y-x}{2\sqrt{2\e y}}\right)+1\right),&x<y\\ (x-y)e^{1-x}-xe^{1-x-y/\e}+\frac{1}{2}e^{1-y}\left(erf\left(\frac{y-x}{2\sqrt{2\e y}}\right)+1\right),&x>y\end{array}\right..$$
        \eenum
\eenum

% Problem 3
\item In class we had examined heat transfer on a flat plate in a uniform stream.  Now consider heat transfer from a cylinder of radius unity and center at the origin, placed in an otherwise uniform stream.  The flow velocity is given by $\bm u = \nabla \phi,$ where the potential $\phi$ is given in polar coordinates as
\begin{equation*}
\phi = \left(r+\frac{1}{r}  \right) \cos \th.
\end{equation*}
The temperature $T$ satisfies the PDE
\begin{equation*}
\bm u \cdot \nabla T = \e \nabla^2 T, \quad r \ge 1,
\end{equation*}
with boundary conditions
\begin{equation*}
T = 1 \;\text{on}\; r=1, \quad T \to 0 \; \text{as}\; r \to \infty.
\end{equation*}
Find the leading-order solution for $0 < \e \ll 1.$  Sketch a graph of the isotherms (lines of constant $T$).  Also, find an expression for $\partial T/\partial r,$ the heat flux, from the cylinder surface.

\n Hint:  Look for a similarity solution of the PDE for the inner problem.\\

Solution:\\

We first derive the PDE governing the heat transference:
$$\nabla\phi\cdot\nabla T=(1-\frac{1}{r^2})\cos\theta T_r-(\frac{1}{r}+\frac{1}{r^3})\sin\theta T_\theta$$
$$\nabla^2 T= T_{rr}+\frac{1}{r}^2 T_{\theta\theta}+\frac{1}{r}T_r$$
$$\implies (1-\frac{1}{r^2})\cos\theta T_r-(\frac{1}{r}+\frac{1}{r^3})\sin\theta T_\theta=\e(T_{rr}+\frac{1}{r}^2 T_{\theta\theta}+\frac{1}{r}T_r).$$
Now we look at the $O(1)$ outer problem by letting $T(r,\theta;\e)\sim T_0(r,\theta)$ and collecting the appropriate terms:
$$(1-\frac{1}{r^2})\cos\theta T_{0r}-(\frac{1}{r}+\frac{1}{r^3})\sin\theta T_{0\theta}=0.$$
This equation has the solution
$$T_0(r,\theta)=f\left(\log\left[\frac{1-r^2}{r}\sin\theta\right]\right).$$
The conditions on $T$ imply that 
$$T_0(\infty,\theta)=f(\infty)=0.$$
We note here that we want the outer solution to satisfy the condition at $\infty$ because having a layer at infinity for gradual heat dissipation would be very strange. We then find that we have a boundary layer that is $O(\sqrt{\e})$ thick along the perimeter of the cylinder in the stream (as the characteristics run right around the cylinder). Thus we let $r=1+\sqrt{\e}\rho$ and $T(r,\theta;\e)=\tau(\rho,\theta;\e)$ and we find the leading order problem in the layer
$$2(\rho\cos\theta \tau_{0\rho}-\sin\theta\tau_{0\theta})=\tau_{0\rho\rho}$$
where $\tau(\rho,\theta;\e)\sim\tau_0(\rho,\theta).$ To solve this PDE we let
$$\tau_0(\rho,\theta)=f(\eta)=f\left(\frac{\rho}{\sqrt{g(\theta)}}\right).$$
This results in the eigenvalue problem
$$2\cos\theta g(\theta)+\sin\theta g'(\theta)=\frac{f''(\eta)}{\eta f'(\eta)}=\lambda,$$
thereby giving us two equations:
$$g'+2\cot\theta g=\lambda\csc\theta,$$
and
$$f''=\lambda\eta f'.$$
The first equation has solution
$$g=c_1\csc^2\theta-\lambda\cot\theta\csc\theta.$$
while the second equation has the solution
$$f=k\sqrt{\frac{\pi}{2|\lambda|}}erf\left(\frac{\sqrt{|\lambda|}\eta}{\sqrt{2}}\right)+c_2.$$
We note that this solution results from choosing $\lambda<0$ to aid in matching later on. We then let $\lambda=-1$ for ease of computation and $c_1=0$ as we are not interested in the homogeneous solution to the $g(\theta)$ equation. This gives the inner solution
$$\tau_0(\rho,\theta)=k\sqrt{\frac{\pi}{2}}erf\left(\frac{\rho\sin\theta}{\sqrt{2\cos\theta}}\right)+c_2.$$
We then apply the boundary condition at $r=1$ to find
$$\tau_0(0,\theta)=c_2=1.$$
Hence the inner solution is
$$\tau_0(\rho,\theta)=k\sqrt{\frac{\pi}{2}}erf\left(\frac{\rho\sin\theta}{\sqrt{2\cos\theta}}\right)+1.$$
To perform the matching, we let $\rho\to \infty$ in the inner solution and $r\to 1$ in the inner solution and find
$$\tau_0\to k\sqrt{\frac{\pi}{2}}+1,$$
$$T_0\to f(-\infty).$$
Therefore we let
$$k=\sqrt{\frac{2}{\pi}}(f(-\infty)-1).$$
The inner solution is then (in terms of the outer variable)
$$\tau_0(r,\theta)=(f(-\infty)-1)erf\left(\frac{(r-1)\sin\theta}{\sqrt{2\e\cos\theta}}\right)+1.$$
The composite solution is then
$$T_c=-f(-\infty)+(f(-\infty)-1)erf\left(\frac{(r-1)\sin\theta}{\sqrt{2\e\cos\theta}}\right)+1+f\left(\log\left[\frac{1-r^2}{r}\sin\theta\right]\right),$$
with the condition that $f(\infty)=0.$ Now to find $\partial T/\partial r$ we simply differentiate the composite solution to find
$$\frac{\partial T}{\partial r}=-\sqrt{\frac{2}{\pi}}\frac{\sin\theta\exp\left[\frac{-(r-1)^2\sin^2\theta}{2\e\cos\theta}\right]}{\sqrt{\e\cos\theta}}+f'\left(\log\left[\frac{1-r^2}{r}\sin\theta\right]\right)\frac{r^2+1}{r^3-r}.$$
I do not understand how to draw the graph of the isotherms, so that part is not included here.





\eenum



\enddocument} 