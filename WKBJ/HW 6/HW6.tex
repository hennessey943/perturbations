\documentclass{article}
\usepackage{amsmath}
\usepackage{amssymb}
\usepackage{bm}
\usepackage{graphicx}
\usepackage{epstopdf}
\DeclareGraphicsRule{.tif}{png}{.png}{`convert #1 `basename #1 .tif`.png}
\usepackage{color}
\pagestyle{plain}
%\pagestyle{empty}
\textheight 9 true in
\textwidth 6.5 true in
\hoffset -.75 true in
\voffset -.75 true in
 
\mathsurround=2pt  \parskip=2pt
\def\crv{\cr\noalign{\vskip7pt}} 
\def\a{\alpha } \def\b{\beta } \def\d{\delta } \def\D{\Delta } \def\e{\epsilon }
\def\g{\gamma } \def\G{\Gamma} \def\k{\kappa} \def\l{\lambda } \def\L{\Lambda }
\def\th{\theta } \def\Th{\Theta} \def\r{\rho} \def\o{\omega} \def\O{\Omega}
\def\ve{\varepsilon} 

\def\sA{{\cal A}} \def\sB{{\cal B}} \def\sC{{\cal C}} \def\sI{{\cal I}}
\def\sR{{\cal R}} \def\sF{{\cal F}} \def\sG{{\cal G}} \def\sM{{\cal M}}
\def\sT{{\cal T}} \def\sH{{\cal H}} \def\sD{{\cal D}} \def\sW{{\cal W}}
\def\sL{{\cal L}} \def\sP{{\cal P}} \def\s{\sigma } \def\S{\Sigma}
\def\sU{{\cal U}} \def\sV{{\cal V}} \def\sY{{\cal Y}}

\def\gm{\gamma -1}
\def\summ{\sum_{j=1}^4}

\def\bb{{\bm b}} \def\yb{{\bm y}}
\def\ub{{\bm u}}  \def\xb{{\bm x}} \def\vb{{\bm v}} \def\wb{{\bm w}}
\def\omegab{{\bm \omega}} \def\rb{{\bm r}} \def\ib{{\bm i}} \def\jb{{\bm j}}
\def\lb{{\bm l}} \def\kb{{\bm k}} \def\Ab{{\bm A}} \def\fb{{\bm f}} \def\Ub{{\bm U}}
\def\Fb{{\bm F}} \def\nb{{\bm n}} \def\Db{{\bm D}} \def\eb{{\bm e}}
\def\gb{{\bm g}}  \def\Gb{{\bm G}} \def\hb{{\bm h}} \def\Yb{{\bm Y}} \def\Rb{{\bm R}} 
\def\Tb{{\bm T}}

\def\As1{{\bf {\cal A}}_1}\def\DO{{\cal D}_0} \def\UO{{\cal U}_0}
\def\ie{{\it{i.e.}}}

\def\ubbar{{\bf {\bar{u}}}} \def\sbar{{\bar{\sigma }}} \def\ubar{{\bar{u}}}  
\def\abar{{\bar{a}}} \def\vbar{{\bar{v}}}  \def\rbar{{\bar{\rho}}}
\def\pbar{{\bar{p}}} \def\ebar{{\bar{e}}} \def\Tbar{{\bar{T}}}
\def\bbar{{\bar{\beta}}} \def\Mbar{{\bar{M}}}  \def \sMbar{{\bar{\cal M}}}
\def\Ebar{{\bar{E}}} \def\sMbar{{\bar{\cal M}}}
\def\sPbar{{\bar{\cal P}}} \def\xbar{{\bar{x}}}

\newcommand{\pdv}[2]{\frac{\partial#1}{\partial#2}}
\newcommand{\dv}[2]{\frac{d#1}{d#2}}
\newcommand{\ord}[2]{#1^{(#2)}}
\newcommand{\vct}[1]{\vec{#1}}

 \newcommand{\bc}{\begin{center}}
 \newcommand{\ec}{\end{center}}
 
 \newcommand{\bq}{\begin{equation}}
 \newcommand{\eq}{\end{equation}}
 
 \newcommand{\beqs}{\begin{eqnarray}}
 \newcommand{\eeqs}{\end{eqnarray}}
 
 \newcommand{\beqa}{\begin{eqnarray*}}
 \newcommand{\eeqa}{\end{eqnarray*}}
 
 \newcommand{\ol}{\overline}
 \newcommand{\ul}{\underline}
 
 \newcommand{\dint}{{\int \!\! \int \!\!}}
 \newcommand{\tint}{{\int \!\! \int \!\! \int \!\!}}
 
 \newcommand{\bfig}{\begin{figure}}
 \newcommand{\efig}{\end{figure}}
 
 \newcommand{\cen}{\centering}
 \newcommand{\n}{\noindent}
 
 \newcommand{\btab}{\begin{table}}
 \newcommand{\etab}{\end{table}}
 
 \newcommand{\btbl}{\begin{tabular}}
 \newcommand{\etbl}{\end{tabular}}
 
 \newcommand{\bdes}{\begin{description}}
 \newcommand{\edes}{\end{description}}
 
 \newcommand{\benum}{\begin{enumerate}}
 \newcommand{\eenum}{\end{enumerate}}
 
 \newcommand{\bite}{\begin{itemize}}
 \newcommand{\eite}{\end{itemize}}
 
 \newcommand{\cle}{\clearpage}
 \newcommand{\npg}{\newpage}
 
 \newcommand{\bss}{\begin{singlespace}}
 \newcommand{\ess}{\end{singlespace}}
 
 \newcommand{\bhalf}{\begin{onehalfspace}}
 \newcommand{\ehalf}{\end{onehalfspace}}
 
 \newcommand{\bds}{\begin{doublespace}}
 \newcommand{\eds}{\end{doublespace}}
 
 \newcommand{\eps}{\mbox{$\epsilon$}} 
 \newcommand{\stilde}{\mbox{$\tilde s$}} 
 \newcommand{\shat}{\mbox{$\hat s$}} 

 \newcommand{\blue}{\color{blue}}
 \newcommand{\red}{\color{red}}
 \newcommand{\magenta}{\color{magenta}}
 \newcommand{\green}{\color{green}}
 \newcommand{\nc}{\normalcolor}





\newcommand{\Ai}{{\rm Ai}}
\newcommand{\Bi}{{\rm Bi}}
\pagestyle{empty}
\begin{document}

\begin{center}
\large{ MATH-6620 \hspace{1in}  PERTURBATION METHODS \hspace{1in}SPRING 2016\\ Homework-6 (OPTIONAL) \\ Assigned Thursday May 5, 2016 \\ Due Monday May 16, 2016.}\end{center}

\bigskip
\n\ul{Michael Hennessey}


\bc {\bf PROBLEMS} \ec

\benum

\item Consider the equation
$$y''+\l^2(x^2-1)y=0,\qquad 0\le x\le 2,$$
as $\l \to \infty$ with $y(0)=0,$ $y'(0)=1.$  First obtain
asymptotic forms for $y(x,\l)$ for $x$ not near the transition
point $x=1.$  Then find the appropriate transition solution, and
hence a leading-order uniformly valid asymptotic solution to the
problem, carefully noting the connection between the three
parts of the solution.\\

Solution:\\

We begin by transforming the equation to move the turning point to 0 by letting $\xi=1-x$, and $y(x;\lambda)=u(\xi,\lambda)$. Then the ODE becomes
$$u''=\lambda^2(2\xi-\xi^2)u\;\;-1\leq \xi\leq 1$$
as $\lambda\to\infty$, with $u(1)=0$ and $u'(1)=-1.$ We then make the standard transformation
$$u(\xi;\lambda)=\phi(\xi)e^{\lambda\o(\xi)}$$
to transform the ODE into
$$\phi''+2\lambda\o'\phi'+\lambda\o''\phi+\lambda^2\o'^2\phi=\lambda^2(2\xi-\xi^2)\phi.$$
We match the powers of $\lambda$ to find
$$\lambda^2:\;\; \o'^2=2\xi-\xi^2\;\implies\; \o=\pm\int\sqrt{2\xi-\xi^2}d\xi.$$
$$\lambda^1:\;\; 2\o'\phi'+\o''\phi=0\;\implies\;\phi=(2\xi-\xi^2)^{-1/4}.$$
Then the leading order asymptotic form of the outer solution is
$$u\sim \left\{\begin{array}{cc}|2\xi-\xi^2|^{-1/4}\left[c_1\cos\left(\lambda\int_\xi^0|2s-s^2|^{1/2}ds\right)+c_2\sin\left(\lambda\int_\xi^0|2s-s^2|^{1/2}ds\right)\right],&-1\leq\xi<0\\ \\
(2\xi-\xi^2)^{-1/4}\left[b_1e^{\lambda\int_0^\xi\sqrt{2s-s^2}ds}+b_2e^{-\lambda\int_0^\xi\sqrt{2s-s^2}ds}\right],&0<\xi\leq 1\end{array}\right..$$
\indent We now determine the solution in the layer at $\xi=0.$ We let $\xi=\lambda^{-p}\zeta$, $(2\xi-\xi^2)\sim2\xi\sim2\lambda^{-p}\zeta$, and $w(\zeta;\lambda)=u(\xi;\lambda).$ Then the ODE becomes
$$w''\lambda^{2p}=2\lambda^{2-p}\zeta w,\;\;-\infty<\zeta<\infty.$$
To match the powers of $\lambda$ on each side of the equation, we take
$$2p=2-p\implies p=\frac{2}{3}.$$
Then we have a near-Airy equation
$$w''=2\zeta w.$$
Here we approximate $w(\zeta;\lambda)\sim\mu_0(\lambda)w_0(\zeta).$ Note this does not change the equation above.
If we take $\eta=B\zeta,$ and $v_0(\eta)=w_0(\zeta)$, we get a new ODE:
$$B^2v_0''=\frac{2\eta}{B}v_0.$$
Then to transform to the Airy equation, we want
$$\frac{2}{B^3}=1\;\implies\;B=2^{1/3}.$$
Thus we have the Airy equation in $v_0(\eta)$ with solution
$$v_0=a_1\Ai(\eta)+a_2\Bi(\eta).$$
We can transform back to $w$ easily:
$$w\sim\mu_0(\lambda)[a_1\Ai(2^{1/3}\zeta)+a_2\Bi(2^{1/3}\zeta)].$$
Now we inspect the asymptotic behavior of the inner and outer solutions in an attempt to match across the turning point. For the inner solution, the asymptotics are known:
$$w\sim\left\{\begin{array}{cc}\mu_0(\lambda)\frac{1}{2^{7/12}\pi^{1/2}|\zeta|^{1/4}}\left[(a_1-a_2)\sin\left(\frac{2\sqrt{2}}{3}|\zeta|^{3/2}\right)+(a_1+a_2)\cos\left(\frac{2\sqrt{2}}{3}|\zeta|^{3/2}\right)\right],&\zeta\to-\infty\\ \\
\mu_0(\lambda)\frac{1}{2^{1/12}\pi^{1/2}\zeta^{1/4}}\left[\frac{a_1}{2}e^{-2\sqrt{2}\zeta^{3/2}/3}+a_2e^{2\sqrt{2}\zeta^{3/2}/3}\right],&\zeta\to\infty\end{array}\right..$$
To look at the outer solution we first make a few approximations. We begin by letting $\xi=\lambda^{-2/3}\zeta.$ Then we approximate $2\xi-\xi^2\sim2\xi\sim2\lambda^{-2/3}\zeta.$ We then take
$$(2\xi-\xi^2)^{-1/4}\sim2^{-1/4}\lambda^{1/6}\zeta^{-1/4},$$
$$\int_\xi^0|2s-s^2|^{1/2}ds\sim\int_{\lambda^{-2/3}\zeta}^0|2s|^{1/2}ds=-\frac{2\sqrt{2}}{3}\lambda^{-1}|\zeta|^{3/2},$$
$$\int_0^\xi\sqrt{2s-s^2}ds\sim\int_0^{\lambda^{-2/3}\zeta}\sqrt{2s}ds=\frac{2\sqrt{2}}{3}\lambda^{-1}\zeta^{3/2}.$$
Hence the asymptotic behavior of the outer solution is:
$$u\sim\left\{\begin{array}{cc}\frac{\lambda^{1/6}}{2^{1/4}|\zeta|^{1/4}}\left[c_1\cos\left(\frac{2\sqrt{2}}{3}|\zeta|^{3/2}\right)-c_2\sin\left(\frac{2\sqrt{2}}{3}|\zeta|^{3/2}\right)\right],&\zeta\to-\infty\\ \\
\frac{\lambda^{1/6}}{2^{1/4}\zeta^{1/4}}\left[b_1e^{2\sqrt{2}\zeta^{3/2}/3}+b_2e^{-2\sqrt{2}\zeta^{3/2}/3}\right],&\zeta\to\infty\end{array}\right..$$
Matching the solutions as $\zeta\to-\infty$ gives
$$\mu_0(\lambda)=2^{1/3}\lambda^{1/6}\pi^{1/2},\;\;\text{ and }\;c_2=a_2-a_1,\;c_1=a_1+a_2.$$
Then using these values, we find from the matching on the other side
$$b_2=\frac{a_1}{\sqrt{2}},\;b_1=\sqrt{2}a_2.$$
We can then express all coefficients in terms of the $b_i$ as
$$c_1=\sqrt{2}b_2+\frac{b_1}{\sqrt{2}},\;c_2=\frac{b_1}{\sqrt{2}}-\sqrt{2}b_2.$$
We now apply the boundary conditions at $\xi=1$ to determine the $b_i$'s.
$$u(1)\sim b_1+b_2=0\implies b_1=-b_2.$$
$$u'(1)\sim 2b_1\lambda=-1\implies b_1=-\frac{1}{2\lambda}.$$
Then the coefficients are
$$b_2=\frac{1}{2\lambda},\;a_1=\frac{\sqrt{2}}{2\lambda},\;a_2=-\frac{\sqrt{2}}{4\lambda},\;c_1=\frac{\sqrt{2}}{4\lambda},\;c_2=-\frac{3\sqrt{2}}{4\lambda}.$$
We can then write our leading-order asymptotic solution as an outer solution
$$u\sim\left\{\begin{array}{cc}|2\xi-\xi^2|^{-1/4}\left[\frac{\sqrt{2}}{4\lambda}\cos\left(\lambda\int_\xi^0|2s-s^2|^{1/2}ds\right)-\frac{3\sqrt{2}}{4\lambda}\sin\left(\lambda\int_\xi^0|2s-s^2|^{1/2}ds\right)\right],&-1\leq\xi<0\\ \\ (2\xi-\xi^2)^{-1/4}\left[-\frac{1}{2\lambda}e^{\lambda\int_0^\xi\sqrt{2s-s^2}ds}+\frac{1}{2\lambda}e^{-\lambda\int_0^\xi\sqrt{2s-s^2}ds}\right],&0<\xi\leq 1\end{array}\right.,$$
and an inner solution
$$u\sim 2^{1/3}\lambda^{1/6}\pi^{1/2}\left[\frac{\sqrt{2}}{2\lambda}\Ai(2^{1/3}\lambda^{2/3}\xi)-\frac{\sqrt{2}}{4\lambda}\Bi(2^{1/3}\lambda^{2/3}\xi)\right],$$
where $\xi$ is in a neighborhood of $0$. We then transform back to the original variable:
$$y\sim\left\{\begin{array}{cc}|1-x^2|^{-1/4}\frac{\sqrt{2}}{4\lambda}\left[\cos\left(\lambda\int_1^x|1-s^2|ds\right)-3\sin\left(\lambda\int_1^x|1-s^2|ds\right)\right],&1<x\leq2\\ \\
(1-x^2)^{-1/4}\frac{1}{2\lambda}\left[e^{-\lambda\int_x^1\sqrt{1-s^2}ds}-e^{\lambda\int_x^1\sqrt{1-s^2}ds}\right],&0\leq x<1\end{array}\right.$$
for the outer solution, and
$$y\sim2^{1/3}\lambda^{1/6}\pi^{1/2}\frac{\sqrt{2}}{2\lambda}\left[\Ai(2^{1/3}\lambda^{2/3}(1-x))-\frac{1}{2}\Bi(2^{1/3}\lambda^{2/3}(1-x))\right],$$
where $x$ is in a neighborhood of 1.


\item The Schr\" odinger equation describing the quantum mechanics of a particle in a potential field $V$ has the form
\begin{equation*}
y''(x) + [E-V(x)]y = 0, \;\; y(\pm \infty) = 0.
\end{equation*}
Take $V(x) = x^4.$  Then $x = \pm E^{1/4}$ are the two turning points.  Find an appropriate expression for the eigenvalues (energy levels) $E_n$ as $n \to \infty,$ for which a nontrivial solution exists.


Solution:\\

We let $x=E^{1/4}\xi,$ and $u(\xi;\lambda)=y(x;\lambda).$ With $V(x)=x^4$, we get the differential equation for $u$:
$$u''+E^{3/2}[1-\xi^4]u=0$$
We then use the WKBJ ansatz, with $\lambda=E^{3/4}$, $u=\phi(\xi)e^{\lambda\o(\xi)},$ to get the differential equation
$$\phi''+\lambda(2\o'\phi'+\o''\phi)+\lambda^2(\o'^2+1-\xi^4)\phi=0.$$
If we look at the powers of $\lambda$ we can solve for $\o$ and $\phi$:
$$\lambda^2:\;\; \o'^2=\xi^4-1\implies \o=\pm\int\sqrt{\xi^4-1}d\xi,$$
$$\lambda^1:\;\; 2\o'\phi'+\o''\phi=0\implies \phi=(\xi^4-1)^{-1/4}.$$
Then $u$ has the general form
$$u=(\xi^4-1)^{-1/4}e^{\pm\lambda\int\sqrt{\xi^4-1}d\xi}.$$
In the region between the two transition points at $\xi=\pm 1$, we argue that our outer solution will have the form
$$u=(1-\xi^4)^{-1/4}\sin\left(\lambda\int\sqrt{1-\xi^4}d\xi+\frac{\pi}{4}\right),$$
as matching in the layer at the turning points will be done with the first Airy function $\Ai$. Then, if we enter the region through the left transition point, $\xi=-1$, we see
$$u=\frac{C}{(1-\xi^4)^{1/4}}\sin\left(\lambda\int_{-1}^\xi\sqrt{1-s^4}ds+\frac{\pi}{4}\right).$$
And if we enter through the right at $\xi=1$, we see
$$u=\frac{D}{(1-\xi^4)^{1/4}}\sin\left(\lambda\int_{\xi}^1\sqrt{1-s^4}ds+\frac{\pi}{4}\right).$$
We rewrite this equation as
$$u=\frac{D}{(1-\xi^4)^{1/4}}\sin\left(\lambda\int_{-1}^1\sqrt{1-s^4}ds+\frac{\pi}{2}-\lambda\int_{\xi}^1\sqrt{1-s^4}ds-\frac{\pi}{4}\right).$$
If we let
$$\theta=\lambda\int_{\xi}^1\sqrt{1-s^4}ds+\frac{\pi}{4},\;Q=\lambda\int_{-1}^1\sqrt{1-s^4}ds+\frac{\pi}{2},$$
then for the solution in the region between the transition points to be equivalent, we enforce
$$C\sin\theta=D\sin(Q-\theta)=-D\sin(\theta-Q).$$
This only occurs when $Q=n\pi$ and $D=(-1)^n C.$ Thus we have
$$\lambda\int_{-1}^1\sqrt{1-s^4}ds+\frac{\pi}{2}=n\pi.$$
Therefore,
$$E_n=\left(\frac{\pi(n-\frac{1}{2})}{\int_{-1}^1\sqrt{1-s^4}ds}\right)^{4/3}.$$


\item Consider the homogeneous ODE
\begin{equation*}
y''(x) - \frac{x}{(x+1)^4} y(x) = 0.
\end{equation*}
Find the first three terms of the asymptotic expansion of each of the two linearly independent solutions for large $x.$\\

Solution:\\

Since our equation is of the form
$$y''+r(x)y=0,$$
we can make the substitution $y=e^{\phi(x)},$ to get the equation
$$\phi''+\phi'^2=\frac{x}{(x+1)^4}.$$
We can approximate this equation by Taylor expanding the term on the right hand side of the equation
$$\phi''+\phi'^2\sim \frac{1}{x^3}-\frac{4}{x^4}+\frac{10}{x^5}.$$
Then, since the leading order term is of the form $x^{-\alpha}$, where $\alpha>2$, we take $\phi\sim\phi_0$ to get
$$\phi_0''\sim\frac{1}{x^3}\implies \phi_0=\frac{1}{2x}+\phi_1.$$
If we substitute this into the ODE for $\phi$, we find
$$\left[\frac{1}{x^3}+\phi_1''\right]+\left(-\frac{1}{2x^2}+\phi_1'\right)^2=\frac{1}{x^3}-\frac{4}{x^4}+\frac{10}{x^5}.$$
Then, expanding the square on the left hand side of the equation and combining the $x^{-4}$ terms, we see
$$\phi_1''\sim-\frac{17}{4x^4}\implies \phi_1=-\frac{17}{24x^2}.$$
Thus our first solution for $y$ has the asymptotic expansion
$$y\sim e^{1/2x}\left[1-\frac{17}{24x^2}+\frac{289}{1152x^4}+...\right].$$
To find the second solution, we try letting $\phi_0=A\ln x.$ Then, substituting this into our ODE for $\phi$, we have
$$-\frac{A}{x^2}+\frac{A^2}{x^2}=\frac{1}{x^3}-\frac{4}{x^4}+\frac{10}{x^5}.$$
Since there are no $x^{-2}$ terms on the right side of the equation, we enforce
$$A^2-A=0\;\implies\; A=0,1.$$
We note that the first solution for $y$ corresponds to the 0 solution here. We take, then, $\phi=\ln(x)+\phi_1.$ Thus our ODE becomes
$$\left[-\frac{1}{x^2}+\phi_1''\right]+\left[\frac{1}{x}+\phi_1'\right]^2=\frac{1}{x^3}-\frac{4}{x^4}+\frac{10}{x^5}.$$
If we make a leading-order approximation for $\phi_1$, we have
$$\phi_1''+\frac{2}{x}\phi_1'\sim\frac{1}{x^3}.$$
Then $\phi_1=-\frac{1}{x}-\frac{\ln{x}}{x}.$
We write
$$\phi=\ln(x)-\frac{1}{x}-\frac{\ln(x)}{x}\;\implies\; y\sim x^{1-1/x}\left[1-\frac{1}{x}+\frac{1}{2x^2}+...\right].$$








\eenum



\enddocument} 